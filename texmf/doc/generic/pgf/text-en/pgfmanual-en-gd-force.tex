% Copyright 2011 by Jannis Pohlmann
%
% This file may be distributed and/or modified
%
% 1. under the LaTeX Project Public License and/or
% 2. under the GNU Free Documentation License.
%
% See the file doc/generic/pgf/licenses/LICENSE for more details.

\section{Graph Drawing Algorithms: Force-Based Methods}
\label{section-library-graphdrawing-force-based}
\label{section-last-graphdrawing-library-in-manual}

{\emph{by Till Tantau and Jannis Pohlmann}}


File status: Jannis has \emph{also} promised me to finally write the
documentation for this file.

\begin{tikzlibrary}{graphdrawing.force}
  Load this package when you wish to use force-based graph drawing
  algorithms. You should load the |graphdrawing| library first.
\end{tikzlibrary}



\subsection{Overview}

Nature creates beautiful graph layouts all the time. Consider a
spider's web: Nodes are connected by edges in a visually most pleasing
manner (if you ignore the spider in the middle). The layout of a
spider's web is created just by the physical forces exerted by the
threads. The idea behind force-based graph drawing algorithms is to
mimic nature: We treat edges as threads that exert forces and simulate
into which configuration the whole graph is ``pulled'' by these
forces.

When you start thinking about for a moment, it turns out that there
are endless variations of the force model. All of these models have
the following in common, however:
\begin{itemize}
\item ``Forces'' pull and push at the nodes in different
  directions.
\item The effect of these forces is simulated by iteratively moving
  all the nodes simultaneously a little in the direction of the forces
  and by then recalculating the forces.
\item The iteration is stopped either after a certain number of
  iterations or when a \emph{global energy minimum} is reached (a very
  scientific way of saying that nothing happens anymore). 
\end{itemize}

The main difference between the different force-based approaches is
how the forces are determined. Here are some ideas what could cause a
force to be exerted between two nodes (and there are more):
\begin{itemize}
\item If the nodes are connected by an edge, one can treat the edge as
  a ``spring'' that has a ``natural spring dimension.'' If the nodes
  are nearer than the spring dimension, they are push apart; if they
  are farther aways than the spring dimension, they are pulled
  together.
\item If two nodes are connected by a path of a certain length, the
  nodes may ``wish to be at a distance proportional to the path
  length''. If they are nearer, they are pushed apart; if they are
  farther, they are pulled together. (This is obviously a
  generalization of the previous idea.)
\item There may be a general force field that pushes nodes apart (an
  electrical field), so that nodes do not tend to ``cluster''.
\item There may be a general force field that pulls nodes together (a
  gravitational field), so that nodes are not too loosely scattered.
\item There may be highly nonlinear forces depending on the distance of
  nodes, so that nodes very near to each get pushed apart strongly,
  but the effect wears of rapidly at a distance. (Such forces are
  known as strong nuclear forces.)
\item There rotational forces caused by the angles between the edges
  leaving a node. Such forces try to create a \emph{perfect angular
  resolution} (a very scientific way of saying that all angles
  at a node are equal).
\end{itemize}

Force-based algorithms combine one or more of the above ideas into a
single algorithm that uses ``good'' formulas for computing the
forces.

Currently, three algorithms are implemented in this library, two of
which are from the first of the following paper, while the third is
from the third paper:


\begin{itemize}
\item
  Y. Hu.
  \newblock Efficient, high-quality force-directed graph drawing.
  \newblock \emph{The Mathematica Journal}, 2006.
\item
  C. Walshaw.
  \newblock A multilevel algorithm for force-directed graph
  drawing.
  \newblock In J. Marks, editor, \emph{Graph Drawing}, Lecture Notes in
  Computer Science, 1984:31--55, 2001. 
\end{itemize}


Our implementation is described in detail in the following
diploma thesis:

\begin{itemize}
\item
  Jannis Pohlmann,
  \newblock \emph{Configurable Graph Drawing Algorithms
    for the \tikzname\ Graphics Description Language,}
  \newblock Diploma Thesis,
  \newblock Institute of Theoretical Computer Science, Univerist\"at
  zu L\"ubeck, 2011.\\[.5em]
  \newblock Online at 
  \url{http://www.tcs.uni-luebeck.de/downloads/papers/2011/2011-configurable-graph-drawing-algorithms-jannis-pohlmann.pdf}
\end{itemize}

In the future, I hope that most, if not all, of the force-based
algorithms become ``just configuration options'' of a general
force-based algorithm similar to the way the modular Sugiyama method
is implemented in the |graphdrawing.layered| library.



\subsection{Controlling and Configuring Force-Based Algorithms}

As indicated in the overview, all force-based algorithms are based on
a general pattern which we detail in the following. Numerous options
can be used to influence the behaviour of this general pattern; more
specific options that apply only to individual algorithms are
explained along with these algorithms.

TODO: Explain the process in general.


\subsubsection{Start Configuration}

TODO: Revise and update text! 



\subsubsection{The Iterative Process and Cooling}

TODO: Revise and update text! 

\begin{key}{/graph drawing/force based/iterations=\meta{number}
  (initially 500)}
  Limits the number of iterations of algorithms for force-based
  layouts to \meta{number}. 

  Depending on the characteristics of the input graph and the parameters
  chosen for the algorithm, minimizing the system energy may require
  many iterations. 

  In these situations it may come in handy to limit the number of
  iterations. This feature can also be useful to draw the same graph
  after different iterations and thereby demonstrate how the spring or
  spring-electrical algorithm improves the drawing step by step.

  The following example shows two drawings generated using two
  different |iteration| limits: 
\begin{codeexample}[]
\tikz \graph [spring layout={iterations=10}]  { subgraph K_n [n=4] };
\tikz \graph [spring layout={iterations=500}] { subgraph K_n [n=4] };
\end{codeexample}

  The same effect happens for all force-based algorithms:
\begin{codeexample}[width=5cm]
\tikz \graph [spring electrical layout={iterations=10}]
  { subgraph K_n [n=4] };
\tikz \graph [spring electrical layout={iterations=500}]
  { subgraph K_n [n=4] };
\end{codeexample}
\end{key}


\begin{key}{/graph drawing/force based/cooling factor=\meta{number}
    (initially 0.95)}
  TODO Jannis: Document this option.
\end{key}

\begin{key}{/graph drawing/force based/convergence tolerance=\meta{number}}
  TODO Jannis: Document this option.
\end{key}

\begin{key}{/graph drawing/force based/initial step dimension=\meta{dimension} (initially 0)}
  TODO Jannis: Document this option.
\end{key}


\subsubsection{Forces and Their Effects: Springs}

TODO: Revise and update text! 

\begin{key}{/graph drawing/force based/spring constant=\meta{number}}
  \keyalias{graph drawing/spring electrical layout}
  TODO Jannis: Document this option.
\end{key}


\begin{key}{/graph drawing/node distance=\meta{dimension} (initially 1cm)}

  Defines the equilibrium length of a spring between two nodes in the
  graph. Both, the |spring layout| and |spring electrical layout|
  algorithms aim at minimizing the energy of the system. The
  natural or equilibrium length of a spring specifies the natural
  distance of any pair of adjacent nodes.

  The |natural spring dimension| option is mostly intended as a scaling
  parameter but it can have other side-effects on the arrangement of
  nodes in the graph as well.

  The following example shows how a simple graph can be scaled by
  changing the |node distance|:
\begin{codeexample}[]
\tikz \graph [spring layout, node distance=7mm] { subgraph C_n[n=3] };
\tikz \graph [spring layout]                    { subgraph C_n[n=3] };
\tikz \graph [spring layout, node distance=15mm]{ subgraph C_n[n=3] };
\end{codeexample}

  The option works in the very same way with |spring electrical layout|
  algorithms:
\begin{codeexample}[]
\tikz \graph [spring electrical layout, node distance=0.7cm] { subgraph C_n[n=3] };
\tikz \graph [spring electrical layout]                      { subgraph C_n[n=3] };
\tikz \graph [spring electrical layout, node distance=1.5cm] { subgraph C_n[n=3] };
\end{codeexample}
\end{key}



\subsubsection{Forces and Their Effects: Electrical Repulsion}



\begin{key}{/graph drawing/force based/electric charge=\meta{number} 
  (default 1)}
  \keyalias{graph drawing/spring electrical layout}

  Defines the electric charge of the node. The stronger the 
  |electric charge| of a node the stronger the repulsion between the
  node and others in the graph. A negative |electric charge| means that
  other nodes are further attracted to the node rather than repulsed,
  although in theory this effect strongly depends on how the 
  |spring electrical layout| algorithm works.

  Two typcal effects of increasing the |electric charge| are distortion
  of symmetries and an upscaling of the drawings:
  \begin{codeexample}[width=5cm] 
\tikz \graph [spring electrical layout, orient=0-1] 
  { 0 [electric charge=1] -- subgraph C_n [n=10] };

\tikz \graph [spring electrical layout, orient=0-1] 
  { 0 [electric charge=5] -- subgraph C_n [n=10] };

\tikz \graph [spring electrical layout, orient=0-1] 
  { [clique] 1 [electric charge=5], 2, 3, 4 };
  \end{codeexample}
\end{key}

\begin{key}{/graph drawing/force based/approximate 
  electric forces=\opt{\meta{boolean}} (default true, initially false)}
  
  Computing the electric forces of the nodes in a graph requires 
  $\mathcal{O}(n^2)$ operations in each iteration of spring- and
  spring-electrical algorithms, where $n$ is the number of nodes in
  the graph. For $l$ coarse graphs, this may increase the runtime of 
  such an algorithm by up to $l\cdot\mathcal{O}(n^2)$ operations. 

  With |approximate electric forces| set to |true|, electric forces 
  are approximated using the Barnes-Hut algorithm known from solving the
  so-called $N$-body problem. This reduces the number of operations
  needed to compute the electric forces to $\mathcal{O}(n\log n)$ per 
  iteration and can thus lead to a significant improvement of the 
  algorithm runtime.
  
  However, this optimization \emph{can} come at the cost of slightly 
  less appealing drawings which is noticable with small graphs in 
  particular. Even though the differences can be very subtle, the
  often reduced quality of the final drawings is the reason why why 
  this feature is turned off by default. Typically, you only want to
  set |approximate electric forces| to |true| to save some time when
  laying out large graphs.

  The following code provides an example where the degraded quality 
  of a drawing with electric force approximation can be noticed:
  \begin{codeexample}[width=5.5cm]
% TODO Jannis: find a better example.

% bad layout due to not computing electric forces accurately
\tikz \graph [spring electrical layout={approximate electric forces}, orient=5-4]
{ subgraph K_n [n=5] };

% better layout thanks to accurate force math
\tikz \graph [spring electrical layout, orient=5-4]
{ subgraph K_n [n=5] };
  \end{codeexample}
\end{key}



\subsubsection{Coarsening}

TODO: Revise and update text! 

\begin{key}{/graph drawing/force based/coarsen=\opt{\meta{boolean}}
  (default true, initially true)}

  Defines whether or not a multilevel approach is used that
  iteratively coarsens the input graph into graphs $G_1,\dots,G_l$ with 
  a smaller and smaller number of nodes. The coarsening stops as soon as
  a minimum number of nodes is reached, as set via the 
  |minimum graph size| option, or if, in the last iteration, the 
  number of nodes was not reduced by at least the ratio specified via 
  |downsize ratio|. 

  A random initial layout is computed for the coarsest graph $G_l$ first.
  Afterwards, it is laid out by computing the attractive and repulsive
  forces between its nodes. 
  
  In the subsequent steps, the previous coarse graph $G_{l-1}$ is 
  restored and its node positions are interpolated from the nodes
  in~$G_l$. The graph $G_{l-1}$ is again laid out by computing the forces
  between its nodes. These steps are repeated with $G_{l-2},\dots,G_1$ until 
  the original input graph $G_0$ has been restored, interpolated 
  and laid out.

  The idea behind this approach is that, by arranging recursively 
  formed supernodes first and then interpolating and arranging their
  subnodes step by step, the algorithm is less likely to settle in a
  local energy minimum (of which there can be many, particularly for
  large graphs). The quality of the drawings with coarsening enabled is
  expected to be higher than graphics where this feature is not applied.

  There are a number of options to fine-tune the coarsening phase.
  They are consolidated in the |/graph drawing/force based/coarsening|
  prefix described below.

  The following example demonstrates how coarsening can improve the
  quality of graph drawings generated with Walshaw's algorihtm 
  |spring electrical layout'|.
  \begin{codeexample}[width=5cm]
\tikz \graph [spring electrical layout'={coarsen=false}, orient=3|4] 
  { 
    { [clique] 1, 2 } -- 3 -- 4 -- { 5, 6, 7 }
  };

\tikz \graph [spring electrical layout'={coarsen}, orient=3|4] 
  { 
    { [clique] 1, 2 } -- 3 -- 4 -- { 5, 6, 7 }
  };
  \end{codeexample}
\end{key}

\begin{key}{/graph drawing/force based/coarsening=\marg{options}}
  Executes the \meta{options} with the path prefix 
  |/graph drawing/force based/coarsening|.

  These options can be used to configure the coarsening approach
  described in the documentation of the 
  |/graph drawing/force based/coarsen| option.
\end{key}

\begin{key}{/graph drawing/force based/coarsening/minimum graph
  size=\meta{number} (initially 2)}
  Defines the minimum number of nodes down to which the graph is 
  coarsened iteratively. The first graph that has a smaller or equal 
  number of nodes becomes the coarsest graph $G_l$, where $l$ is the 
  number of coarsening steps. The algorithm proceeds with the steps 
  described in the documentation of the |coarsen| option.

  In the following example the same graph is coarsened down to two
  and four nodes, respectively. The layout of the original graph is 
  interpolated from the random initial layout and is not improved
  further because the forces are not computed (0 iterations). Thus, 
  in the two graphs, the nodes are placed at exactly two and four
  coordinates in the final drawing:
  \begin{codeexample}[width=5.5cm]
\tikz \graph [spring layout={
                iterations=0,
                coarsening={minimum graph size=2}
              }] 
{ subgraph C_n [n=8] };

\tikz \graph [spring layout={
                iterations=0,
                coarsening={minimum graph size=4}
              }] 
{ subgraph C_n [n=8] };
  \end{codeexample}
\end{key}

\begin{key}{/graph drawing/force based/coarsening/downsize
  ratio=\meta{number} (initially 0.25)}
  Minimum ratio between 0 and 1 by which the number of nodes between 
  two coarse graphs $G_i$ and $G_{i+1}$ need to be reduced in order for 
  the coarsening to stop and for the algorithm to use $G_{i+1}$ as the 
  coarsest graph $G_l$. Aside from the input graph, the optimal value 
  of |downsize ratio| mostly depends on the coarsening scheme being
  used. Possible schemes are |collapse independent edges| and 
  |connect independent nodes| which are explained later in this
  document.

  Increasing this option possibly reduces the number of coarse
  graphs computed during the coarsening phase as coarsening will stop as
  soon as a coarse graph does not reduce the number of nodes
  substantially. This may speed up the algorithm but if the size of the 
  coarsest graph $G_l$ is much larger than |minimum graph size|, the 
  multilevel approach may not produce drawings as good as with a lower
  |downsize ratio|.
  \begin{codeexample}[width=5cm]
% 1. ratio too high, coarsening stops early, benefits are lost
\tikz \graph [spring electrical layout'={
                coarsening={downsize ratio=1.0},
              }, node distance=7mm, orient=3|4] 
  { 
    { [clique] 1, 2 } -- 3 -- 4 -- { 5, 6, 7 }
  };

% 2. ratio set to default, coarsening benefits are visible
\tikz \graph [spring electrical layout'={
                coarsening={downsize ratio=0.2},
              }, node distance=7mm, orient=3|4] 
  { 
    { [clique] 1, 2 } -- 3 -- 4 -- { 5, 6, 7 }
  };
  \end{codeexample}
\end{key}

\begin{key}{/graph drawing/force based/coarsening/collapse independent
  edges=\opt{\meta{boolean}} (default true, initially true)}
  During the coarsening phase, the number of nodes in the graph is
  repeatedly reduced using a coarsening scheme such as 
  |collapse independent edges| or |connect independent nodes|.

  If the scheme |collapse independent edges| is enabled, which happens
  to be the default setting, coarse versions of the input graph are 
  generated by finding a maximal independent edge set and by collapsing
  the edges from this set. Nodes adjacent to each of these edges are
  merged into supernodes by which they are replaced in the next coarse
  version of the graph. Edges and nodes that are not related to the
  maximum independent edge set are maintained in the new graph.

  Collapsing the edges of a maximum independent edge set reduces the
  number of nodes of the graph by up to but never more than 50\,\%. 
  This means that the upper limit for reasonable values of 
  |downsize ratio| to be used in combination with 
  |collapse independent edges| is $0.5$. 
  
  Compared to |connected independent nodes| this is the less aggressive
  coarsening scheme and typically yields better drawings.
\end{key}

%\begin{key}{/graph drawing/force based/coarsening/connected independent nodes=\opt{\meta{boolean}} (default true, initially
%    false)}
%  TODO Jannis: Implement and document this option.
%\end{key}



\subsection{Spring Layouts}


\begin{gdalgorithm}{spring layout}{Hu2006 spring}
  The spring algorithm described in the paper:

  \begin{itemize}
  \item
    Y. Hu.
    \newblock Efficient, high-quality force-directed graph drawing.
    \newblock \emph{The Mathematica Journal}, 2006.
  \end{itemize}

  ``Efficient, High-Quality
    Force-Directed Graph Drawing,'' \emph{The Mathematica Journal,}
    2006 by Yifan Hu. 
\begin{codeexample}[]
\tikz \graph [spring layout] { a -> {b,c -> {d,e,f} } -> h -> a };    
\end{codeexample}
  TODO Jannis: Describe the algorithm!
\end{gdalgorithm}


\subsection{Spring-Electrical Layouts}


\begin{gdalgorithm}{spring electrical layout}{Hu 2006 Spring Electrical}
  The spring-electrical algorithm described in the paper:
  \begin{itemize}
  \item
    Y. Hu.
    \newblock Efficient, high-quality force-directed graph drawing.
    \newblock \emph{The Mathematica Journal}, 2006.
  \end{itemize}
\begin{codeexample}[]
\tikz \graph [spring electrical layout] { a -> {b,c -> {d,e,f} } -> h -> a };    
\end{codeexample}
\end{gdalgorithm}

\begin{gdalgorithm}{spring electrical layout'}{Walshaw 2000 Spring Electrical}
  The spring-electrical algorithm described in the paper:
  \begin{itemize}
  \item
    C. Walshaw.
    \newblock A multilevel algorithm for force-directed graph
    drawing.
    \newblock In J. Marks, editor, \emph{Graph Drawing}, Lecture Notes in
    Computer Science, 1984:31--55, 2001. 
  \end{itemize}
\begin{codeexample}[]
\tikz \graph [spring electrical layout'] { a -> {b,c -> {d,e,f} } -> h -> a };    
\end{codeexample}
  TODO Jannis: Describe the algorithm!
\end{gdalgorithm}


\endinput

%%% Local Variables: 
%%% mode: latex
%%% TeX-master: "pgfmanual-pdftex-version"
%%% End: 
